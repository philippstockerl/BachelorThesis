%=======================
% Bachelor's Thesis Proposal – Philipp Stockerl
%========================
\documentclass[11pt,a4paper]{article}

% --- Packages ---
\usepackage[english]{babel}
\usepackage[T1]{fontenc}
\usepackage[utf8]{inputenc}
\usepackage{lmodern}
\usepackage{geometry}
\geometry{margin=2.5cm}
\usepackage{graphicx}
\usepackage{titlesec}
\usepackage{hyperref}
\usepackage{enumitem}
\usepackage{ulem}
\normalem
\usepackage{xcolor}

% --- Metadata ---
\newcommand{\studentname}{Philipp Stockerl}
\newcommand{\matnr}{107829}
\newcommand{\chair}{Chair of Business Decisions \& Data Science}
\newcommand{\professor}{Prof.\ Dr.\ Marc Goerigk}
\newcommand{\faculty}{Faculty of Business Administration and Economics}
\newcommand{\unilogo}{figs/PassauLogo.png}
\newcommand{\thesistype}{Bachelor's Thesis Proposal}
\newcommand{\proposaltitle}{Static and Dynamic Vehicle Routing under Uncertainty}

% --- Hyperref colors ---
\hypersetup{
	colorlinks=true,
	linkcolor=black,
	urlcolor=blue!50!black,
	citecolor=black
}

% --- Section formatting ---
\titleformat{\section}{\large\bfseries}{\thesection}{0.6em}{\MakeUppercase}
\titlespacing*{\section}{0pt}{5.0ex plus .2ex}{0.6ex}
\setlength{\parskip}{6pt}
\setlength{\parindent}{0pt}

% --- Header ---
\newcommand{\makeproposalhead}{%
	\noindent
	\begin{minipage}[t]{0.62\textwidth}
		{\large \textbf{\studentname}}\\[0.2em]
		\textit{Matr. No} \ \matnr\\[0.8em]
		\textbf{\chair}\\
		\professor
	\end{minipage}%
	\hfill
	\begin{minipage}[t]{0.35\textwidth}
		\raggedleft
		\vspace{0pt}
		\includegraphics[height=1.7cm]{\unilogo}\\[-0.2em]
		{\small \faculty}
	\end{minipage}
	
	\vspace{1.0em}
	\hrule height 0.8pt
	\vspace{0.9em}
	
	{\Large \textbf{\thesistype}: \textbf{\proposaltitle}}\\[-0.3em]
	\vspace{-1.5em}
}

\begin{document}
	
	\makeproposalhead
	
	\section*{Research Question}
	How can static and dynamic optimization approaches be compared in vehicle routing problems when operating under uncertain and spatially variable environmental conditions?
	
	\section*{Motivation and Background}
	During my studies, I have focused on Operations Research and Decision Analytics within both the Business Administration and Information Systems degree programs. 
	I completed all courses previously offered by the former Chair of Prof.\ Dr.\ Alena Otto, including the bachelor seminar on uncertainty in competitive facility location, with the original intention of writing my thesis under her supervision. 
	Following her unexpected departure from the University of Passau in early 2024 and the subsequent written notice regarding my application, I now aim to continue my thesis project in close alignment with her research area under the supervision of Prof.\ Dr.\ Marc Goerigk at the Chair of Business Decisions \& Data Science.
	
	In my seminar paper with Prof.\ Dr.\ Alena Otto on Competitive Facility Location, I studied Drezner’s model (\textit{Competitive Facility: Random Utility}) and explored classical approaches such as Huff and Hotelling. 
	In the \textit{Practical Course: Management Science}, I implemented a \textit{Minimum Cost Flow Problem} using Gurobi and Python, which has prepared me well to handle both the robust extension and its implementation in this thesis. 
	
	The deployment of Unmanned Aerial Vehicles (UAVs) in logistics has increased rapidly. 
	Efficient route planning is critical for safety, energy consumption, and reliability. 
	While static optimization techniques compute fixed routes based on predicted data, dynamic approaches allow continuous adaptation as new information becomes available. 
	This thesis examines how both paradigms perform when confronted with uncertain or imperfect information about travel conditions, highlighting the trade-off between conservatism and adaptability in UAV routing.
	
		\section*{Connection to Business Informatics and \\Information Systems}
		This thesis is closely related to the field of Business Informatics and Information Systems, as it combines methods of Operations Research with data-driven decision support and intelligent information systems. It follows the principles of model-driven decision support systems as outlined by \textit{Power and Sharda (2007)}, in which analytical models constitute the reasoning core of digital decision environments. 
		
		The integration of robust optimization and dynamic replanning modeled into a simulation-based architecture reflects how prescriptive analytics can be embedded into decision-support workflows that process uncertain and spatially distributed data. The framework thereby represents an experimental instance of a model-driven decision support system (DSS) that dynamically reacts to information updates. 
		
		The dynamic component builds on the perspective of \textit{Ulmer, Heilig, and Voß (2017)}, who emphasize the value and challenges of real-time information in dynamic dispatching contexts. Their findings motivate the analysis of how real-time updates affect both the reliability and efficiency of adaptive routing systems.
		
		Finally, the system-oriented perspective of \textit{Lacomme, Rault, and Sevaux (2021)} provides an architectural parallel, illustrating how Operations Research methods can be operationalized through REST-based decision support systems such as the Mapotempo platform. This highlights the practical relevance of linking optimization-based models with information-system architectures, reinforcing the interdisciplinary character of this thesis within the domain of Business Informatics.
	
	
	\section*{Methodology and Data}
	The study compares representative static and dynamic optimization models in a controlled simulation environment. 
	Synthetic spatial networks will be generated where edge costs are derived from spatial random fields (SRF) that mimic environmental effects such as wind or turbulence. 
	Two data layers will be created:
	\begin{itemize}[leftmargin=1.2em]
		\item A \textbf{blurred forecast map} representing anticipated cost patterns (e.g., weather prediction). 
		\item A \textbf{clear actual map} representing true realized travel conditions as encountered by the vehicle.
	\end{itemize}
	The static model, based on the Bertsimas Sim robust optimization framework, computes a pre-planned route using the forecast data and an uncertainty budget~$\Gamma$. 
	This route is then provided as the initial path to a dynamic replanning model implemented with algorithms like D*~Lite, which receives local sensor updates and may adapt the route if actual conditions differ. 
	Instead of binary obstacles, uncertain regions will be modeled as \textit{delay zones} that increase traversal cost without blocking connectivity.
	
	\section*{Evaluation Design}
	The performance of both methods will be analyzed under varying relationships between forecast and actual conditions:
	\begin{itemize}[leftmargin=1.2em]
		\item Actual conditions are \textbf{better than expected}: robust model overcompensates; dynamic method may improve efficiency.
		\item Actual conditions \textbf{match the forecast}: both models expected to behave similarly.
		\item Actual conditions are \textbf{worse than expected}: stress-test for robustness and recovery capability.
	\end{itemize}
	Key metrics include total realized cost, computational time, robustness ratio (worst/nominal cost), and adaptability (number of successful replans). 
	This setup enables a quantitative assessment of how pre-computed robustness and real-time adaptability balance efficiency and reliability in uncertain routing environments.
	
	\section*{Selected References: Information Systems/ \\Wirtschaftsinformatik}
		\begin{itemize}[leftmargin=1.2em]
			\item Power, D. J., \& Sharda, R. (2007). \textit{Model-driven decision support systems: Concepts and research directions.} \textbf{Decision Support Systems}, 43(3), 1044–1061. \\https://doi.org/10.1016/j.dss.2005.05.030
			
			\item Lacomme, P., Rault, G., \& Sevaux, M. (2021). \textit{Integrated decision support system for rich vehicle routing problems.} \textbf{Expert Systems with Applications}, 178, 114998. \\https://doi.org/10.1016/j.eswa.2021.114998
			
			\item Ulmer, M. W., Heilig, L., \& Voß, S. (2017). \textit{On the value and challenge of real-time information in dynamic dispatching of service vehicles.} \textbf{Business \& Information Systems Engineering}, 59(3), 161–171. https://doi.org/10.1007/s12599-017-0468-2
		\end{itemize}
		
		
	\section*{Selected References: Management Science/ Operations Research}
		\begin{itemize}
			\item Koenig, S., \& Likhachev, M. (2002). \textit{D* Lite.} In Proceedings of the AAAI Conference on Artificial Intelligence (pp. 476–483). AAAI Press.
			
			\item Liu, Y., Li, J., Sun S. \& Yu B. (2019). \textit{Advances in Gaussian random field generation: A review.} Computational Geosciences, 23, 1011-1047. https://doi.org/10.1007/s10596-019-09867-y
			
			\item Bertsimas, D., \& Sim, M. (2003). \textit{Robust discrete optimization and network flows.} Mathematical Programming, 98(1), 49–71. https://doi.org/10.1007/s10107-003-0396-4
			
			\item Goerigk, M., \& Hartisch, M. (2024). \textit{An Introduction to Robust Combinatorial Optimization: Concepts, Models and Algorithms for Decision Making under Uncertainty.} International Series in Operations Research \& Management Science, Vol. 361. Springer, Cham. https://doi.org/10.1007/978-3-031-61261-9
			
		\end{itemize}
		
	
\end{document}
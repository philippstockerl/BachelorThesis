%=======================
% Bachelor's Thesis Proposal – Philipp Stockerl
%========================
\documentclass[11pt,a4paper]{article}

% --- Packages ---
\usepackage[english]{babel}
\usepackage[T1]{fontenc}
\usepackage[utf8]{inputenc}
\usepackage{lmodern}
\usepackage{geometry}
\geometry{margin=2.5cm}
\usepackage{graphicx}
\usepackage{titlesec}
\usepackage{hyperref}
\usepackage{enumitem}
\usepackage{ulem}
\normalem
\usepackage{xcolor}
\usepackage{amsmath, amssymb}
\usepackage{longtable}
\usepackage{booktabs}

% --- Metadata ---
\newcommand{\studentname}{Philipp Stockerl}
\newcommand{\matnr}{107829}
\newcommand{\chair}{Chair of Business Decisions \& Data Science}
\newcommand{\professor}{Prof.\ Dr.\ Marc Goerigk}
\newcommand{\faculty}{Faculty of Business Administration and Economics}
\newcommand{\unilogo}{figs/PassauLogo.png}
\newcommand{\thesistype}{Bachelor's Thesis}
\newcommand{\proposaltitle}{Static and Dynamic Vehicle Routing under Uncertainty}

% --- Hyperref colors ---
\hypersetup{
	colorlinks=true,
	linkcolor=black,
	urlcolor=blue!50!black,
	citecolor=black
}

% --- Section formatting ---
\titleformat{\section}{\large\bfseries}{\thesection}{0.6em}{\MakeUppercase}
\titlespacing*{\section}{0pt}{5.0ex plus .2ex}{0.6ex}
\setlength{\parskip}{6pt}
\setlength{\parindent}{0pt}

% --- Header ---
\newcommand{\makeproposalhead}{%
	\noindent
	\begin{minipage}[t]{0.62\textwidth}
		{\large \textbf{\studentname}}\\[0.2em]
		\textit{Matr. No} \ \matnr\\[0.8em]
		\textbf{\chair}\\
		\professor
	\end{minipage}%
	\hfill
	\begin{minipage}[t]{0.35\textwidth}
		\raggedleft
		\vspace{0pt}
		\includegraphics[height=1.7cm]{\unilogo}\\[-0.2em]
		{\small \faculty}
	\end{minipage}
	
	\vspace{1.0em}
	\hrule height 0.8pt
	\vspace{0.9em}
	
	{\Large \textbf{\thesistype}: \textbf{\proposaltitle}}\\[-0.3em]
	\vspace{-1.5em}
}

\begin{document}
	
	\makeproposalhead
	
	
	\section{Spatial Random Field Generation}
	
		\subsection{Covariance Models}
		
		\begin{itemize}
			\item Gaussian Kernel:\\
			\[
			\gamma(r) = \sigma^2 \left( 1 - \exp\!\left( -\left( s \cdot \frac{r}{\ell} \right)^{2} \right) \right) + n,
			\qquad
			s = \frac{\sqrt{\pi}}{2}.
			\]
			
			\item Exponential Kernel:
			
			\item Matern Kernel:
			
			\item Stable Kernel:
			
			\item Rational Kernel:
			
		\end{itemize}
		
		
		
		\subsection{Fast Fourier Transformation Generation}
		
		
		
		
		
		
	\newpage
	\newpage
	\section*{List of Symbols and Abbreviations}
	
		
		
		
		
		
		
		
\newpage
\section{Robust Static Network-Flow Model under Budget Uncertainty}

This section introduces the mathematical formulation of the robust shortest-path model under budgeted uncertainty, following the approach of \textit{Bertsimas \& Sim (2003)}. 
The model determines a cost-minimal path between a start and goal node in a directed network while accounting for uncertain edge costs within a predefined uncertainty budget~$\Gamma$. 
The formulation extends the standard single-commodity network-flow model by integrating a robust optimization component, allowing a controlled level of conservatism against cost deviations.
\\
% ============================
\subsection{Mathematical Formulation}

\begin{align*}
	\min_{x \in \mathcal{X}} \max_{c \in \mathcal{U}} \sum_{e \in E} c_e x_e 
	&& \text{(4.8)} \nonumber \\[0.3em]
	= \min \sum_{e \in E} \hat{c}_e x_e + \Gamma \pi + \sum_{e \in E} \rho_e 
	&& \text{(4.9)} \nonumber \\[0.3em]
	\text{s.t. } \pi + \rho_e \ge d_e x_e, \quad \forall e \in E 
	&& \text{(4.10)} \nonumber \\[0.3em]
	\sum_{e \in \delta^+(v)} x_e - \sum_{e \in \delta^-(v)} x_e = b_v, \quad \forall v \in V 
	&& \text{(4.11)} \nonumber \\[0.3em]
	x_e \in \{0,1\}, \quad \pi \ge 0, \quad \rho_e \ge 0, \quad \forall e \in E 
	&& \text{(4.12)--(4.14)}
\end{align*}
\\
% ============================
\subsection{Model Description}

\paragraph{Objective Function (4.8)} 
Equation~(4.8) defines the robust shortest-path problem as a nested min–max optimization problem. 
The outer minimization selects a feasible path~$x$ from the feasible set~$\mathcal{X}$, while the inner maximization represents nature’s adversarial choice of edge cost realizations~$c_e$ within the uncertainty set~$\mathcal{U}$.

\paragraph{Deterministic Equivalent (4.9)} 
Using the \textit{Bertsimas–Sim} budgeted uncertainty model, the inner maximization can be reformulated as a linear deterministic equivalent. 
Here, $\hat{c}_e$ denotes the nominal cost of edge~$e$, $d_e$ its maximum deviation, and $\Gamma$ the uncertainty budget defining how many edge costs may simultaneously deviate to their worst case. 
The auxiliary variables $\pi$ and $\rho_e$ represent the global and edge-specific deviation protections, respectively.

\paragraph{Key Robust Constraints (4.10)} 
This constraint ensures that each selected edge’s deviation~$d_e$ is adequately covered by the global uncertainty variable~$\pi$ or by its corresponding slack variable~$\rho_e$. 
This limits the model’s overall conservatism while ensuring robustness against up to~$\Gamma$ worst-case edge deviations.

\paragraph{Flow Balance Constraints (4.11)} 
Constraint~(4.11) enforces flow conservation in the directed graph~$G = (V, E)$. 
For each node~$v \in V$, the difference between outgoing and incoming flows equals the node balance parameter~$b_v$, which takes the value~$1$ for the source node,~$-1$ for the sink node, and~$0$ for all intermediate nodes. 
This guarantees that exactly one continuous path connects the start and goal nodes.

\paragraph{Feasibility and Nonnegativity (4.12–4.14)} 
The binary variable~$x_e$ indicates whether edge~$e$ is included in the selected path ($x_e = 1$) or not ($x_e = 0$). 
The continuous nonnegative variables~$\pi$ and~$\rho_e$ ensure valid uncertainty protections, maintaining feasibility and nonnegativity in the robust formulation.
\\
% ============================
\subsection{Symbols and Parameters}

\begin{table}[h!]
	\centering
	\renewcommand{\arraystretch}{1.2}
	\begin{tabular}{lll}
		\hline
		\textbf{Symbol} & \textbf{Description} & \textbf{Type / Domain} \\ \hline
		$\mathcal{X}$ & Feasible set of flow-conserving paths & Set \\
		$\mathcal{U}$ & Uncertainty set for edge costs & Set \\
		$E$ & Set of edges in the directed network & Set \\
		$V$ & Set of nodes in the directed network & Set \\
		$e \in E$ & Edge index, $e = (u,v)$ connecting nodes $u$ and $v$ & Index \\
		$\delta^+(v)$, $\delta^-(v)$ & Sets of outgoing and incoming edges for node $v$ & Sets \\
		$b_v$ & Node supply/demand balance ($1$, $-1$, or $0$) & Parameter \\[0.3em]
		$\hat{c}_e$ & Nominal cost of edge $e$ & Parameter \\
		$d_e$ & Maximum cost deviation of edge $e$ & Parameter \\
		$\Gamma$ & Uncertainty budget controlling robustness & Parameter \\[0.3em]
		$x_e$ & Binary decision variable for edge $e$ & Decision variable ($x_e \in \{0,1\}$) \\
		$\pi$ & Global deviation threshold variable & Decision variable ($\pi \ge 0$) \\
		$\rho_e$ & Local slack variable for edge $e$ & Decision variable ($\rho_e \ge 0$) \\[0.3em]
		$c_e$ & Realized edge cost ($c_e \in \mathcal{U}$) & Derived quantity \\
		$\sum_{e \in E} c_e x_e$ & Total path cost & Expression \\
		\hline
	\end{tabular}
	\caption{Symbols and parameters used in the robust static network-flow model.}
	\label{tab:model_symbols}
\end{table}
% ============================
\subsection{Implementation Correspondence}

The model is implemented in Python using the Gurobi optimization solver. 
Each binary variable~$x_e$ corresponds to an edge~$e = (u,v)$ in the directed graph. 
Flow balance constraints~(4.11) are constructed for each node~$v \in V$ using the outgoing and incoming edge sets~$\delta^+(v)$ and~$\delta^-(v)$. 
The robust budget constraints~(4.10) link the binary edge decisions~$x_e$ with the uncertainty control variables~$\pi$ and~$\rho_e$, allowing a trade-off between nominal efficiency and robustness. 
The objective function~(4.9) minimizes the sum of the nominal edge costs and the uncertainty surcharge term~$\Gamma \pi + \sum_{e \in E} \rho_e$. 
This linearized deterministic equivalent enables the model to be solved efficiently as a mixed-integer linear program (MILP), providing the optimal robust path for a given uncertainty budget~$\Gamma$.
	

	
	
	
	
	

	
	
	
	
	
	
	
	
	
	
	
	
	
	
	
	
	
	
	
	
	
	
	
	
	
	\newpage
	\section{Dynamic Path Replanning}
	
	
	
	
	
	
	
	
	
	\newpage
	
	
	
	
	
	
	
	
	
	
	
	
	
	
	
	
	
	
	

\end{document}